\section{Experiment and Analysis}

This section presents a series of experiments to validate the proposed method in terms of physical realism and performance scalability. We also perform a comparative analysis against existing common approximation methods to evaluate its applicability and advantages in real-world scenarios.

\subsection*{Experiment 1: Validation of Physical Realism}

\subsubsection*{Objective}
To verify whether the proposed method can accurately simulate the floating, sinking, and rotational behaviors of rigid bodies in fluids, with a particular focus on the correctness of torque effects.

\subsubsection*{Setup}
\begin{itemize}
		\item \itemheader{Platform} Unity 2022.3.29f1, Windows 11, 13th Gen Intel(R) Core(TM) i9-13900HX @ 2.20GHz, 16GB RAM.
		\item \itemheader{Scene} A lightweight plank freely falls into a static water body.
		\item \itemheader{Physical Settings}
		\begin{itemize}
				\item Water density $\rho = 1.33$.
				\item Plank density is lower than water to ensure floating.
				\item Energy dissipation terms are disabled to purely observe buoyancy effects.
		\end{itemize}
		\item \itemheader{Sampling Density} 200 surface samples per surface area.
\end{itemize}

\subsubsection*{Measurement Metrics}
\begin{itemize}
		\item Variation of the plank's center of mass height over time.
		\item Variation of the plank's lowest point height over time.
		\item Variation of the pitch angle of the plank's main surface normal over time.
\end{itemize}

\subsubsection*{Results}

\begin{figure}[H]
	\centering
	\scalebox{1}{
		\begin{tikzpicture}
			\begin{axis}[
				width=5in, height=2in,
				enlargelimits=false,
				xlabel={time (seconds)},
				ylabel={meter},
				ymin=-2, ymax=5, ytick={-2,-1,...,5},
				axis y line*=left,
			]
				\addplot[black] coordinates { (0,0) (5,0) };
				\addplot[olive] table[x=t, y=com] {../Thesis/figures/simulation-record.dat};
				\addplot[blue] table[x=t, y=min] {../Thesis/figures/simulation-record.dat};
			\end{axis}
			\begin{axis}[
				width=5in, height=2in,
				enlargelimits=false,
				ylabel={degree},
				ymin=0, ymax=90, ytick={0,15,...,90},
				axis y line*=right,
				ylabel near ticks, yticklabel pos=right,
			]
				\addplot[color=red] table[x=t, y=rx] {../Thesis/figures/simulation-record.dat};
			\end{axis}
			\matrix [draw, below left, fill=white] at (4.25in, 1.25in) {
				\node[olive, font=\footnotesize] {center-of-mass}; \\
				\node[blue, font=\footnotesize] {lowest point}; \\
				\node[red, font=\footnotesize] {pitch}; \\
			};
		\end{tikzpicture}
	}
	\label{fig:exp1}
\end{figure}

\subsubsection*{Analysis}

\begin{itemize}
		\item The plank quickly experiences localized buoyant forces upon contacting the water and starts rotating, eventually stabilizing with its large face facing upward.
		\item The center of mass height exhibits a typical damped oscillation, consistent with real-world floating behavior.
\end{itemize}

\subsubsection*{Summary}
The experimental results confirm that the proposed method can effectively simulate buoyancy and induced rotational behaviors, with dynamics aligning with physical expectations.

\subsection*{Experiment 2: Performance and Scalability Test}

\subsubsection*{Objective}
To test how the method impacts performance (FPS) as the number of submerged objects increases, and to evaluate load capacity under different sampling densities.

\subsubsection*{Setup}
\begin{itemize}
		\item \textbf{Same platform and water environment}.
		\item \itemheader{Scene} Lightweight blocks are continuously generated and dropped at a fixed rate.
		\item \itemheader{Comparative Variables}
		\begin{itemize}
				\item With water (buoyancy simulation enabled) vs. without water (baseline FPS).
				\item Different surface sampling densities ($s$): 20, 50, 100 per object per surface area.
		\end{itemize}
\end{itemize}

\subsubsection*{Measurement Metrics}
\begin{itemize}
		\item FPS versus number of submerged objects.
		\item (Optional) Average per-frame buoyancy computation time.
\end{itemize}

\subsubsection*{Results}

\begin{figure}[H]
	\centering
	\begin{tikzpicture}
		\begin{axis}[
			width=6in, height=4in,
			enlargelimits=false,
			xlabel={Number of objects},
			xmin=0, xmax=100,
			ymin=0, ymax=200,
			ylabel=FPS,
			legend style={at={(1,0)}, anchor=south east, legend columns=1}
		]
			\addplot[black] table[x index=0, y index=1, col sep=comma] {./figures/exp2/d0.log};
			\addlegendentry{simulation off}

			\addplot[red] table[x index=0, y index=1, col sep=comma] {./figures/exp2/d10.log};
			\addlegendentry{$s=10$}

			\addplot[blue] table[x index=0, y index=1, col sep=comma] {./figures/exp2/d20.log};
			\addlegendentry{$s=20$}

			\addplot[green] table[x index=0, y index=1, col sep=comma] {./figures/exp2/d50.log};
			\addlegendentry{$s=50$}

			\addplot[purple] table[x index=0, y index=1, col sep=comma] {./figures/exp2/d100.log};
			\addlegendentry{$s=100$}
		\end{axis}
	\end{tikzpicture}
	\label{fig:exp2}
\end{figure}

\subsubsection*{Analysis}

\begin{itemize}
		\item FPS decreases linearly with the number of submerged objects, consistent with $O(N)$ complexity expectations.
		\item Under moderate sampling densities (20–50 samples/m$^2$), real-time performance (above 60 FPS) is maintained with up to 30–50 objects.
\end{itemize}

\subsubsection*{Summary}
The method demonstrates good scalability for moderate numbers of objects and allows flexible trade-offs between accuracy and performance by adjusting sampling density.

\subsection*{Experiment 3: Comparative Analysis}

\subsubsection*{Objective}
To further validate the advantages of the proposed method, a comparison is made against a common manual buoyancy approximation method using multiple sphere pontoons.

\subsubsection*{Design}
\begin{itemize}
		\item Same plank object and water environment.
		\item Comparative Metrics:
		\begin{itemize}
				\item Simulation realism (e.g., reasonable rotation behavior).
				\item Buoyancy error (using stable submerged depth as a reference).
				\item FPS variation trend.
		\end{itemize}
\end{itemize}

\subsubsection*{Results}

{\LARGE\color{red}TODO}

\subsubsection*{Analysis}

{\LARGE\color{red}TODO}

\subsection*{Summary}

\begin{itemize}
		\item The proposed method outperforms traditional approximation methods in terms of physical realism and torque capturing.
		\item It maintains good scalability in real-time applications involving small to moderate numbers of submerged objects.
		\item It offers greater flexibility and generality when adapting to complex or procedurally generated geometries.
\end{itemize}