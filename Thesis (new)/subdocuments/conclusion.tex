\section{Conclusion and Future Work}

This paper presents a Monte Carlo-based surface sampling method for real-time buoyancy estimation. Designed for buoyancy simulation needs in real-time applications, the proposed framework achieves physical effect computation without manual configuration, adapts to arbitrary complex geometries, and allows adjustable trade-offs between precision and performance.

Through systematic experiments, the method demonstrates excellent performance in physical realism, scalability, and application flexibility. It effectively supports the dynamic interaction and response of floating objects with water bodies in common real-time development scenarios. The current method meets the initial goals of real-time simulation, complex geometry adaptability, and workflow simplification, offering a practical and easily integrable solution for water interaction physics simulation.

Future work can extend in the following directions:

\begin{itemize}
	\item First, the current implementation models local water pressure using a hydrostatic approximation. For cases involving intense flow or high-speed object motion, deviations may occur. Future exploration could involve introducing local fluid dynamics correction models to further improve simulation accuracy in dynamic water environments while maintaining real-time performance.

	\item Second, the current method focuses on one-way coupling, considering only the fluid’s influence on solids. Future work could integrate simplified fluid solvers to incorporate feedback effects from solid motion onto the water body, thereby enabling more realistic two-way fluid-solid interaction simulations. Such an extension is expected to require new fluid dynamics modules and careful balancing between computational resource demands and simulation stability.

	\item Additionally, since the computations between surface samples are independent, the method exhibits good parallelism. Future optimization based on GPU acceleration or multi-core parallel frameworks could significantly enhance the sampling and integration process, allowing for higher sampling densities and larger numbers of floating objects while maintaining real-time performance in more complex scenes.
\end{itemize}

Although these directions extend beyond the initial scope of this study, they provide promising paths for further improving the precision and expressiveness of real-time water physics simulation.
