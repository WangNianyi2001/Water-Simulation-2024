\section{Introduction}

Buoyancy simulation plays a crucial role in interactive physical environments such as video games and virtual training systems.
Accurately capturing how rigid bodies interact with fluids is essential for achieving realism in scenes involving floating, sinking, or partially submerged objects.
However, integrating such simulations into real-time applications remains a nontrivial challenge.

One primary difficulty lies in the need to repeatedly query an object’s geometry to determine which parts are submerged, in what orientation, and with what surface normals---information that is typically expensive to compute at runtime.
While modern game engines provide powerful physics and rendering systems, they often lack efficient, high-level abstractions for interacting with geometry at the fidelity required for fluid-solid interactions.
Implementing these queries from scratch is possible, but performance constraints make naive approaches infeasible for real-time usage.

Existing buoyancy simulation methods often rely on heuristics or proxy approximations that sacrifice accuracy or require extensive manual setup per object.
To avoid this, we present a Monte-Carlo-based method that addresses these trade-offs by directly approximating buoyancy forces through surface sampling.
Our approach enables a practical compromise between accuracy and computational cost by adjusting the sampling density dynamically.
It could naturally capture both linear and rotational effects (torque), and is applicable to arbitrary rigid body shapes without the need for object-specific preprocessing.

We also provided a way to generalize the core structure of our method onto any physical effect governed by surface or volume integrals.
With minimal integration effort, this method can serve as a drop-in component in modern game engines, enabling physically plausible, real-time field-object interaction at scale.