\section{Introduction}

In real-time applications like video games, it is sometimes necessary to simulate the motion of objects submerged in water.
With regular methods, making frequent queries to the geometrical shape of the simulated object is inevitable to ensure the realism of the simulation.
Such queries could go on at a frequency of many times per object per frame.
The underlying calculation to perform such simulation would require the geometry information of the target objects, but common game engines don't provide well-encapsulated interfaces for it due to the consideration of runtime performance.
It'd be easy to run into performance issues if one were to implement these queries from scratch.

In this article, we propose a method to simulate based on the Monte-Carlo method.
It is possible to change the sampling intensity to balance between simulation quality and performance.
With some effort, we could transform the core idea of this method into a generic framework that is capable for simulating any kind of physical effect of rigid bodies moving through a force field.