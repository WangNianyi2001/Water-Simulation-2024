\section{Related Works}

Speaking in the field of water simulation, the exisiting methods could be divided into two parties: the Lagrangian method, which models water into individual particles and simulates the physical behavior of each, and the Eulerian method, which treats the entire water body as a continuous region and represents it as a field \cite{GOU09}.
Most of the researches on these methods focus on simulating either water itself or the two-way physical coupling between water and submerged objects;
but in this research, we mainly focus on the one-way effect caused on the objects by the field.
The Lagrangian method would have little contribution to our goal, so we will scope on only the Eulerian method.

\cite{teng2016eulerian} proposed a algorithm entirely based on the Eulerian approach, for simulating the two-way coupling between deformable solids and incompressible fluid.
This algorithm could achieve the goal of recreating the realism well, but is too slow for real-time applications, as it would take seconds for a single frame to be rendered.

\cite{GER13} and \cite{BAJ20} proposed algorithms for real-time simulation, but both are based on proxy-based approximation of different manners.
\cite{GER13} takes the entire fluid volume as the induced field of a
limited amount of "panels"---essentially generating seeds at certain location that represent the surrounding fluid motion in their adjacent spaces.
It is cheap enough to run at a real-time scale in a 2D game, but sacrifices the accuracy from reducing the freedom of the fluid’s behavior.
\cite{BAJ20} disects the submerged objects into inter-connected virtual containers.
When an object is submerged in water, the algorithm would simulate the gradual change of water floating into the containers, thus creating a sinking effect.
This algorithm is also cheap enough to be run in real-time, but requires all objects that would be submerged to be equipped with manually-configured virtual container setup.

There were two Monte-Carlo fluid simulation algorithm (\cite{sugimoto2024velocity} and \cite{rioux2022monte}) proposed in recent years, but they are all fluid solvers, not capable for handling the physical effect on objects moving through.