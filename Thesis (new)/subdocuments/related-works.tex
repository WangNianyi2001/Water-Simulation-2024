\section{Related Works}

The simulation methods for fluid-solid interactions can generally be divided into two categories: two-way coupling and one-way coupling. Two-way coupling aims to simultaneously simulate both the influence of the fluid on the solid and the feedback effect of the solid on the fluid \cite{benra2011comparison}. It is commonly employed in high-precision fluid-solid interaction simulations. In recent years, some studies have proposed GPU-based real-time two-way coupling methods, particularly for particle fluids interacting with deformable solids, by combining meshless fluid solvers with finite element structural simulations \cite{yang2012realtime}. However, such methods often rely heavily on high-performance parallel hardware and are overly complex for the specific goal of rigid-body buoyancy estimation.

In contrast, one-way coupling only computes the influence of the fluid on the solid, ignoring the feedback effect from the solid to the fluid. This significantly reduces computational cost and is therefore more suitable for real-time buoyancy simulation \cite{benra2011comparison}. Our research focuses on rigid-body buoyancy under one-way coupling, rather than full fluid-solid two-way interactions.

Within the one-way coupling framework, various works have explored buoyancy estimation methods based on either volume integration or surface integration \cite{BAJ20}\cite{kellomaki2014rigid}. Among these, surface sampling methods are particularly efficient for complex rigid body models. By applying the divergence theorem, the volume integration problem can be converted into a surface integration problem, greatly reducing computational effort and providing a theoretical foundation for surface sampling approaches \cite{UTSAWikiDivergenceTheorem}.

Real-time buoyancy estimation often adopts various heuristic approximations, such as dividing an object into multiple simple shapes (e.g., boxes or spheres), computing their buoyancy separately, and aggregating the results \cite{GER13}\cite{BAJ20}. While these methods are fast, they require extensive manual setup and tend to suffer from reduced accuracy when handling complex or dynamically changing object shapes.

In industry, mainstream game engines offer different levels of built-in buoyancy support. Unreal Engine provides an official ``Water Buoyancy Component'' as part of its Water Plugin, which approximates floating objects using multiple spherical pontoons \cite{UnrealWaterBuoyancyComponent}. This allows lightweight buoyancy simulation suitable for most gameplay needs but can struggle with detailed torque feedback or complex partial submersion effects. In contrast, Unity does not offer an official buoyancy component. However, community resources, such as the Unite 2015 talk ``A Little Math for Your Big Ideas'', have discussed basic techniques for implementing buoyancy manually \cite{Unite2015ALittle}. As a result, developers using Unity often need to implement their own buoyancy systems from scratch.

Monte Carlo methods have been primarily used in fluid simulation itself, such as handling particle distributions or high-dimensional velocity field integrations \cite{rioux2022monte}\cite{sugimoto2024velocity}. Applying Monte Carlo concepts to rigid-body buoyancy estimation, especially under real-time one-way coupling scenarios, remains a relatively underexplored direction.

Overall, real-time buoyancy simulation requires carefully balancing accuracy and performance \cite{liu2021faster}. While traditional volume integration approaches are theoretically more precise, they are too costly for real-time applications. In contrast, the Monte Carlo surface sampling method proposed in this research offers flexible sampling density adjustment, achieving a practical compromise between physical plausibility and computational efficiency.