\section{Method}

\subsection{Motivation}

Estimating the buoyant force and torque acting on rigid bodies in real-time applications poses significant challenges, especially when balancing physical realism and computational efficiency. Existing real-time methods often approximate buoyancy using simplified proxy objects, such as collections of spheres or boxes, which require manual setup and tend to degrade in accuracy for complex or dynamic geometries. More rigorous approaches based on volume integration or two-way fluid-solid coupling provide higher accuracy but are computationally infeasible for interactive environments.

In static fluids, the buoyant force on a submerged body can be derived directly by integrating the hydrostatic pressure over the wetted surface. Therefore, a surface integral formulation offers a physically correct and computationally tractable foundation for buoyancy estimation, without the need for explicit volumetric modeling or full fluid simulation.

However, accurately evaluating the surface integral in real time is nontrivial for general geometries. Direct quadrature would require dense sampling of the object’s surface, leading to excessive computational cost. To overcome this, we propose a Monte-Carlo-based surface sampling framework that approximates the buoyancy integral by aggregating contributions from randomly selected surface points, enabling adaptive trade-offs between accuracy and performance.

\subsection{A Monte-Carlo Framework}

To approximate the surface integral in real time, we randomly sample a set of points over the submerged portion of the rigid body's surface. Each sample point is treated as representing a localized surface element. For simplicity and generality, we assume access to an abstract uniform surface sampling mechanism without delving into mesh-level implementation details.

At each sampled point, the local fluid pressure is computed based on the depth below the fluid surface, using a hydrostatic approximation. Additionally, to better reflect dynamic behaviors, we include form drag and viscous drag forces, proportional to the normal and tangential components of the surface-relative velocity, respectively. The total local force at each sample is the sum of these contributions, scaled appropriately by empirical coefficients that control drag intensities.

To sum up the local contributions without losing torque information, they need to be converted from force-position pairs into force-torque pairs that acts directly on the center-of-mass of rigid bodies. The conversion is given by simple Newtonian physics:
\begin{equation}
	(f, x) \mapsto (f, x\times f),
\end{equation}
where $x$ is the arm from the center-of-mass to the point, and $f$ is the force acting on the point. They can then be added together member-wisely:
\begin{equation}
	(f_1,\tau_1)+(f_2,\tau_2)=(f_1+f_2,\tau_1+\tau_2).
\end{equation}

The net buoyancy force and torque are them estimated by aggregating all local contributions. Since the random sampling process does not guarantee even coverage of the true surface area, we introduce a normalization step. The estimated total force \(\hat{F}\) is given by:
\begin{equation}
	\hat{F}=\frac{A}{\sum_i w_i} \sum_{i}w_i(f_i, x_i\times f_i),
\end{equation}
where \(A\) is the known submerged surface area, \(w_i\) is the representative weight (area) of each sample, and \(f_i\) is the force computed at sample \(x_i\). This normalization ensures that the Monte Carlo estimator remains unbiased and robust across varying sampling densities.

Compared to traditional proxy-based methods that rely on simplified primitives and manual configuration, our method provides greater automation and generality, especially for complex or procedurally generated geometries. In contrast to full two-way coupling schemes that require solving fluid dynamics explicitly, our one-way coupling approach targets static or quasi-static fluid environments, achieving significant computational savings without compromising key physical realism. Furthermore, by leveraging stochastic sampling, the method allows for dynamic adjustment of sampling density, offering developers a flexible trade-off between simulation fidelity and runtime performance in real time applications.