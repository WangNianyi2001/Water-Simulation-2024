\section{Conclusion and Discussion}

Starting from digging into Archimede's law, we managed to build a realistic model of the buoyancy force on solid bodies submerged in fluid.
To overcome the difficulties on discretization, we then apply the Monte-Carlo method on the model.
By doing so, the model yields into a practical algorithm that can freely balance between quiality and performance.
To make it more realistic, we introduced the resistance terms into the model.
After generalizing, the model can even support introducing custom physical contributions.
In the last, we put the model on a run and verified its feasibility.
The result shows that the model is successful.

\paragraph{Generalization}

Although our model is proposed mainly based on the assumption of a static water body,
the key formula (\ref{net-water-force}) is, however, completely irrelevant to this assumption.
It is fundamentally a Monte-Carlo method to simulate the effect of any fleid on a rigidbody.
One should be able to easily substitute the $\mathbf{f}_{\text{pressure}}$ term with their own version to make it compatible with a more complex implementation of the water body, for example, a particle or height field based water system; or substitute the entire $\mathbf{f}_{\text{all}}$ term to simulate other kinds of fields like a wind blow or magnetic field.

\paragraph{Limitations}

Our model only provides a solution to one half of fluid-solid coupling -- how solid objects move in fluid.
It is unable to use our model to simulate the backward effect that the solid object causes to the fluid.

\paragraph{Further Work}

Under a low sample rate, the model might result in unstablility.
The random nature of the model would unevitably introduce noise into the physical outcome, leading to a Brownian-like vibrating behavior.
This could appear artifactual, especially when the submerged body is close to the stable state.

To avoid artifacts, one way is to increase the sample rate,
but it would increase the cost on performance;
another way is to use stabilizing or filtering algorithms.
Doing so might introduce delay in the physical response;
but depends on the actual application, the downstream developers shall be able to choose the best choice on their own.

\section*{Postscripts and Acknowledgements}

This is the first time I have ever written a serious academical article.
It is guaranteed that there will be naive mistakes all over the place.
Please excuse me, thou reader.
If thou hast spotted any mistake, please feel free to contact me at \url{wangnianyi2001@outlook.com}.
My apologies in advance!

Thanks to the team of my graduation project, \emph{Nani Core} (\url{https://github.com/nani-core}).
The idea of this article rose when I was making the water system in the project.
They established the possibilty for this article to happen.
Special thanks to 陈恩晖 (Omnisch) and 张嘉玥 (Limko).
They are two really, really reliable co-workers and good friends of mine.
They have given me the greatest mental and physical support on the project, and an unforgetful memory in my graduation year.

\section*{Appendix}

A sample project of the simulation experiment is published at \url{https://github.com/WangNianyi2001/Water-Simulation-2024}.