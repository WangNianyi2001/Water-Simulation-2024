\section{Introducing the Resistance Forces}

In reality, objects moving through fluid would gradually lose their kinetic energy due to the resistance between its surface and the water body.
If this mechanism is missing, then an object in water might oscillate forever and cannot tend to a still state, which is artifactual.
Also, our method is discretized, meaning that there might be floating-point precision errors introduced within every calculation steps.
If not dealt with, these errors could potentially cause the physical system's energy to continuously accumulate and thus breaking the energy conservation law.
Therefore, we would like this energy-losing mechanism to be added into our model by introducing some resistance components.

The total resistance force of a solid object moving fluid, formally called \emph{parasitic drag}, is a combination of two separate sources of force:
\begin{itemize}
	\item The form drag, or pressure drag, due to the size and shape of the object.
	\item The skin friction drag, or viscous drag, due to the friction between the fluid and the surface of the object.
\end{itemize}

\subsection{Form Drag}

The form drag is caused by the surface of an object being pushed by the fluid blocking on its moving path.
It is proportional to the square of the relative velocity between the moving object and the fluid.
The formula is given as (\ref{form-drag}); the letter $\mathbf{D}$ is used to represent a drag force.

\begin{equation}
	\mathbf{D}_{\text{form}}=\frac{1}{2}\,\rho\,u^2\,c_f\,A,
	\label{form-drag}
\end{equation}
where:
\begin{itemize}
	\item $\rho$ is the mass density of the fluid;
	\item $u$ is the flow velocity relative to the object;
	\item $c_f$ is the drag coefficient, a dimensionless coefficient that depends on the fluid's properties;
	\item $A$ is the area of the object's cross section perpendicular to $u$.
\end{itemize}

This formula is only suitable for calculating a rough net drag force for a whole object;
for our application, we need to adapt it into a form that can be calculated on the sample points.
This could be rather easily done by ``translating'' the terms in (\ref{form-drag}).
\begin{itemize}
	\item $\rho$ remains the same.
	\item $u$ would be the velocity of the body \emph{at the sample point}.
		The position matters because the body might be rotating; in which case the world velocity of each point would differ.
	\item $c_f$ remains the same.
	\item $A$ would be the perpendicular area of the triangular face that the sample lies on.
		This can be calculated by multiplying the real area of the face with the dot product of its normal and the unit vector long $u$.
\end{itemize}

The form drag on each sample point is given as formula (\ref{sample-form-drag}).
\begin{equation}
	\mathbf{d}_{\text{form}}=\frac{1}{2}\,\rho\,\mathbf{v}^2\,c_f\,|\text{area}|\,(\hat{\mathbf{v}}\cdot\mathbf{n}).
	\label{sample-form-drag}
\end{equation}

One important note is that if a face is \emph{not} repelling the fluid in front of it when moving, it should not produce any form drag.
This can be checked by the dot product during the calculation of $A$: if the product is positive, simply drop the drag or make it zero.

\subsection{Viscous Drag}

The viscous drag is caused by the friction between the fluid and the object's surface as the surfaces passes along \emph{parallelly}.
Similar to the form drag, the viscous drag is also proportional to the square of the flow velocity.
The rough formula of a net viscous drag is given as formula (\ref{viscous-drag}).
\begin{equation}
	\mathbf{D}_{\text{viscous}}=\int_{\partial V}\frac{1}{2}\,\rho\,\mathbf{v}^2\,c_v\,\text{d}A.
	\label{viscous-drag}
\end{equation}
It might looks different from (\ref{form-drag}), but really it is the same thing written under the notation of integral.

The formula for the viscous drag on each sample point can be copied from (\ref{sample-form-drag}), but the coefficient needs to be changed, and the dot product should be replaced by the ``parallelliness'' between the face and the flow velocity, which can be given by the absolute value of the cross product.
The formula is given as (\ref{sample-viscous-drag}).
\begin{equation}
	\mathbf{d}_{\text{viscous}}=\frac{1}{2}\,\rho\,\mathbf{v}^2\,c_v\,|\text{area}|\,|\hat{\mathbf{v}}\times\mathbf{n}|.
	\label{sample-viscous-drag}
\end{equation}
If the method needs to be applied on an Euclidean space at a dimension other than 3 where a cross product is not well-defined, one could substitude the cross product term by $\sqrt{1-(\hat{\mathbf{v}}\cdot\mathbf{n})^2}$.

\subsection{Integrating Into the Model}

To integrate the resistance forces into our model, we can substitute the sample force term in (\ref{net-pressure-force}) to make it contain the new forces.
The new formula is given as (\ref{net-water-force}).
\begin{equation}\begin{split}
	\mathbf{F}_{\text{all}}=\frac
		{
			\bigoplus_{s}
			\left(
				w(s)\cdot\mathbf{f}_{\text{all}}(s)
				,
				\text{arm}(s)
			\right)
		}
		{\sum_{s}w(s)},
	\\
	\mathbf{f}_{\text{all}}=\mathbf{f}_{\text{pressure}}+\mathbf{d}_{\text{form}}+\mathbf{d}_{\text{viscous}}.
	\label{net-water-force}
\end{split}\end{equation}

Lastly, to ultimately suppress any potential unrealistic energy outbreak, such as a body bouncing back into the sky, one could forcefully apply a energy dissipation of any objects submerged in fluid.
This could be achieved by applying a force and a torque in the opposite direction of the object's velocity and angular velocity continuously in every frame.
The intensity of the force could be controlled by an configurable coefficient.